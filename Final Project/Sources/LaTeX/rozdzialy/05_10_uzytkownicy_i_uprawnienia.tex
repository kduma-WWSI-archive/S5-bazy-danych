\subsection{Użytkownicy, role i uprawnienia}

Przewidziane zostały 3 role dla użytkowników:
\begin{itemize}
	\item \texttt{admini\_systemu} - Administratorzy - zarządzają obiektami, kategoriami i miastami
	\item \texttt{operatorzy\_systemu} - Operatorzy - zarządzają najmami (wynajem i zwroty)
	\item \texttt{uzytkownicy\_systemu} - Użytkownicy - zarządzają swoimi danymi oraz najmami
\end{itemize}
Do tych ról przypisane zostały uprawnienia zgodnie z tabelą \ref{table:permissions}.

Dodatkowo fabrycznie zostali utworzeni dwaj użytkownicy:
\begin{itemize}
	\item \texttt{admin} z hasłem \texttt{yourStrong(!)Password} przypisany do roli \texttt{admini\_systemu}
	\item \texttt{operator} z hasłem \texttt{yourStrong(!)Password} przypisany do roli \texttt{operatorzy\_systemu}
\end{itemize}

Kolejnych użytkowników roli \texttt{uzytkownicy\_systemu}, można tworzyć za pomocą procedury składowanej z listingu \ref{lst:procedura-utworz_uzytkownika-przyklad} - \texttt{utworz\_uzytkownika}.

\begin{table}[t]
	{\renewcommand{\arraystretch}{1.5}
		\begin{tabu} to \textwidth { |X[2,l]||X[1,c]|X[1,c]|X[1,c]| }
			\hline
			& \textbf{Administratorzy} & \textbf{Operatorzy} & \textbf{Użytkownicy} \\
			\hline
			\hline
			\multicolumn{4}{|c|}{\textbf{Tabele}} \\
			\hline
			\texttt{kategorie} & S, I, U, D & S & S \\
			\hline
			\texttt{miasta} & S, I, U, D & S & S \\
			\hline
			\texttt{dzielnice} & S, I, U, D & S & S \\
			\hline
			\texttt{obiekty} & S, I, U, D & S & S \\
			\hline
			\texttt{uzytkownicy} & S & S & S\footnotemark, U\footnotemark \\
			\hline
			\texttt{najmy} & S, I, U & S, I, U & S\footnotemark, I, U\footnotemark \\
			\hline
			\multicolumn{4}{|c|}{\textbf{Widoki}} \\
			\hline
			\texttt{lista\_najmow} & S & S & - \\
			\hline
			\texttt{lista\_niepopularnych\_obiektow} & S & - & - \\
			\hline
			\texttt{lista\_popularnosci\_obiektow} & S & - & - \\
			\hline
			\multicolumn{4}{|c|}{\textbf{Procedury}} \\
			\hline
			\texttt{utworz\_uzytkownika} & X & X & - \\
			\hline
			\multicolumn{4}{|c|}{S - SELECT, I - INSERT, U - UPDATE, X - EXECUTE} \\
			\hline
		\end{tabu}
		\label{table:permissions}
		\caption{Role i ich uprawnienia}
	}
\end{table}

\addtocounter{footnote}{-4}
\stepcounter{footnote} \footnotetext{wybieranie i edycja ograniczone tylko do swoich danych}
\stepcounter{footnote} \footnotetext{tylko pola nazwisko, imie, wiek, adres, telefon, plec}
\stepcounter{footnote} \footnotetext{wybieranie i edycja ograniczone tylko do swoich danych}
\stepcounter{footnote} \footnotetext{tylko pole data\_zakonczenia}


