\subsection{Tabele}

Wszystkie tabele są tworzone przez 13 skryptów \texttt{SQL}:
\begin{itemize}
	\item \href{run:Sources/SQL/1. Tabele/001_Utworzenie_tabeli_z_miastami.sql}{Utworzenie tabeli z miastami}
	\item \href{run:Sources/SQL/1. Tabele/002_Utworzenie_tabeli_z_dzielnicami.sql}{Utworzenie tabeli z dzielnicami}
	\item \href{run:Sources/SQL/1. Tabele/003_Utworzenie_relacji_pomiedzy_miastami_a_dzielnicami.sql}{Utworzenie relacji pomiedzy miastami a dzielnicami}
	\item \href{run:Sources/SQL/1. Tabele/004_Utworzenie_tabeli_z_kategoriami.sql}{Utworzenie tabeli z kategoriami}
	\item \href{run:Sources/SQL/1. Tabele/005_Utworzenie_tabeli_z_obiektami.sql}{Utworzenie tabeli z obiektami}
	\item \href{run:Sources/SQL/1. Tabele/006_Utworzenie_relacji_pomiedzy_dzielnicami_a_obiektami.sql}{Utworzenie relacji pomiedzy dzielnicami a obiektami}
	\item \href{run:Sources/SQL/1. Tabele/007_Utworzenie_relacji_pomiedzy_kategoriami_a_obiektami.sql}{Utworzenie relacji pomiedzy kategoriami a obiektami}
	\item \href{run:Sources/SQL/1. Tabele/008_Utworzenie_tabeli_z_uzytkownikami.sql}{Utworzenie tabeli z uzytkownikami}
	\item \href{run:Sources/SQL/1. Tabele/009_Utworzenie_tabeli_z_najmami.sql}{Utworzenie tabeli z najmami}
	\item \href{run:Sources/SQL/1. Tabele/010_Utworzenie_indeksu_unikatowego_w_tabeli_z_najemcami.sql}{Utworzenie indeksu unikatowego w tabeli z najemcami}
	\item \href{run:Sources/SQL/1. Tabele/011_Utworzenie_relacji_pomiedzy_uzytkownikami_a_najmami.sql}{Utworzenie relacji pomiedzy uzytkownikami a najmami}
	\item \href{run:Sources/SQL/1. Tabele/012_Utworzenie_relacji_pomiedzy_obiektami_a_najmami.sql}{Utworzenie relacji pomiedzy obiektami a najmami}
	\item \href{run:Sources/SQL/1. Tabele/013_Utworzenie_indeksu_unikatowego_w_tabeli_z_uzytkownikami}{Utworzenie indeksu unikatowego w tabeli z uzytkownikami}
\end{itemize}

Tworzenie relacji pomiędzy tabelami oraz indeksów zostało oddzielone od operacji tworzenia poszczególnych tabel - celem tego działania jest lepsza organizacja skryptów. Dodatkowo oddzielając te operacje, w przypadku wystąpienia jakiegoś błędu jesteśmy w stanie określić co i gdzie się "wysypało".