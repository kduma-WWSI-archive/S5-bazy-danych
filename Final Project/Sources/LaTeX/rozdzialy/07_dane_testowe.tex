\section{Dane Testowe}

\subsection{Skrypt tworzący dane testowe}

W skrypcie tworzącym, w pierwszej kolejności definiujemy zbiór danych testowych (na przykład tak jak na listingu \ref{lst:create-data-script-declare}), a w dalszej części tworzymy i dodajemy do bazy danych, dane testowe (listing \ref{lst:create-data-script} lub listing \ref{lst:create-data-script-2}).

Część tabel, jak na przykład tabela \texttt{uzytkownicy}, jest wypełniana w sposób losowy\footnote{Zastanawiający może być warunek \texttt{WHERE d.id IS NOT NULL AND k.id IS NOT NULL} - jest to pewnego rodzaju "obejście" procesu optymalizacji SQL Server'a. Gdybyśmy nie zastosowali takiej struktury to dane były by wylosowane tylko za pierwszym razem - czyli wszystkie wiersze miały by taką samą wartość danego pola, a w tym przypadku takie zachowanie jest zachowaniem niepożądanym.} (Listing \ref{lst:create-data-script-2}). Oznacza to że wraz z każdym uruchomieniem skryptu dane dodane do bazy danych będą inne.

\lstinputlisting[language=SQL, caption=Fragment deklaracji danych testowych, firstline=6, lastline=8, label={lst:create-data-script-declare}]{../../skrypt_tworzacy_dane_testowe.sql}

\lstinputlisting[language=SQL, caption=Fragment prostego tworzenia danych testowych, firstline=36, lastline=42, label={lst:create-data-script}]{../../skrypt_tworzacy_dane_testowe.sql}

\lstinputlisting[language=SQL, caption=Fragment losowego tworzenia danych testowych, firstline=96, lastline=107, label={lst:create-data-script-2}]{../../skrypt_tworzacy_dane_testowe.sql}

\subsubsection{Wynik uruchomienia skryptu tworzącego dane testowe w trybie wsadowym}

W momencie wykonywania skryptu, podaje on liczbę dodanych wierszy do każdej z tabel, w formacie takim jak na listingu \ref{lst:create-data}.

\lstinputlisting[caption=Wynik uruchomienia całego skryptu tworzącego dane testowe w trybie wsadowym, label={lst:create-data}]{../Logs/skrypt_tworzacy_dane_testowe.sql-second-run.log}

\subsection{Skrypt usuwający dane testowe}

Ponieważ nasze tabele posiadają kolumny autonumerowanie (typu \texttt{IDENTITY}), to nie możemy skasować danych przy pomocy funkcji \texttt{TRUNCATE}. Aby uzyskać podobny wynik (opróżnienie tabeli) wykorzystuję dyrektywę \texttt{DBCC CHECKIDENT(<tabela>, RESEED, 0)} która powoduje zresetowanie autonumerowania w tabeli.

\lstinputlisting[language=SQL, caption=Fragment skryptu usuwającego dane testowe, firstline=1, lastline=2, label={lst:destroy-data-script}]{../../skrypt_usuwajacy_dane_testowe.sql}

\subsubsection{Wynik uruchomienia skryptu usuwającego dane testowe w trybie wsadowym}

\lstinputlisting[caption=Wynik uruchomienia całego skryptu usuwającego dane testowe w trybie wsadowym, label={lst:destroy-data}]{../Logs/skrypt_usuwajacy_dane_testowe.sql.log}