\section{Skrypt tworzący dane testowe}

W skrypcie tworzącym, w pierwszej kolejności definiujemy zbiór danych testowych (na przykład tak jak na listingu \ref{lst:create-data-script-declare}), a w dalszej części tworzymy i dodajemy do bazy danych, dane testowe (na przykład tak jak na listingu \ref{lst:create-data-script} lub listingu \ref{lst:create-data-script-2}).

Część tabel, jak na przykład tabela \texttt{uzytkownicy}, jest wypełniana w sposób losowy (Listing \ref{lst:create-data-script-2}). Oznacza to że wraz z każdym uruchomieniem skryptu dane dodane do baziy danych będą inne.

\lstinputlisting[language=SQL, caption=Fragment deklaracji danych testowych, firstline=6, lastline=8, label={lst:create-data-script-declare}]{../../skrypt_tworzacy_dane_testowe.sql}

\lstinputlisting[language=SQL, caption=Fragment prostego tworzenia danych testowych, firstline=53, lastline=60, label={lst:create-data-script}]{../../skrypt_tworzacy_dane_testowe.sql}

\lstinputlisting[language=SQL, caption=Fragment losowego tworzenia danych testowych, firstline=70, lastline=82, label={lst:create-data-script-2}]{../../skrypt_tworzacy_dane_testowe.sql}

\subsection{Wynik uruchomienia całego skryptu w trybie wsadowym}

W momencie wykonywania skryptu, podaje on liczbę dodanych wierszy do każdej z tabel, w formacie takim jak na listingu \ref{lst:create-data}.

\lstinputlisting[caption=Wynik uruchomienia całego skryptu tworzącego dane testowe w trybie wsadowym, label={lst:create-data}]{../Logs/skrypt_tworzacy_dane_testowe.sql-second-run.log}